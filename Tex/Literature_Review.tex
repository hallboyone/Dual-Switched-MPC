 \section{Literature Review}
 \subsection{Key contributions of this paper}
 This paper examines systems that
 \begin{enumerate}
 	\item are distributed,
 	\item are coupled through states,
 	\item are uncoupled through constraints,
 	\item have locally switching dynamics,
 	\item have a switching network according to local switches,
 	\item have local, constrained, external switching signals.
 \end{enumerate}
 How does this compare with other papers?

\cite{Ahandani2020} examines decentralized systems under MPC with switching communication topologies and dynamics coupling. The switching is generated from a centralized signal with a minimum dwell time constraint. The local agents are assumed LTI while the cross dynamics change according to the switching signal. This leads to an external disturbance drawn from a switching set applied to each node. This is rejected using tube-based, switched MPC. 

\cite{Monasterios2019} examines systems under MPC that are provided a preview of future disturbances. The preview evolves through disturbed shifting dynamics. The primary results of the paper show that, if the change in disturbance is ``small enough'', the system can maintain feasibility and stability with augmented terminal control elements. It does not, however, consider switching either locally nor remotely. 

\cite{Danielson2019} looks at LTI systems with a single external switching signal that can alter the system's dynamics, constraints, and objectives. It scales poorly, however.

The authors of \cite{Li2020} look at distributed MPC with switching communication topologies. Each agent shares its predicted state trajectory with its neighbors and must stay ``close'' to the trajectory it shared at the last time step. By balancing how much the trajectory may change, the cost of a trajectory change, and terminal constraints, leads to stability. This work does not consider state constraints beyond terminal constraints and does not consider switching in the local dynamics, only the communication network. 