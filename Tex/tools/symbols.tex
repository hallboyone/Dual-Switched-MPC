% ========================== Basic ==========================
\newcommand{\dyn}[1]{\ensuremath{#1^x}}        % Related to system dynamics
\newcommand{\dst}[1]{\ensuremath{#1^\sigma}}   % Related to system disturbance 

% =========================== Sets ==========================
\renewcommand{\int}{\ensuremath{\mathbb{Z}}}
\newcommand{\real}{\ensuremath{\mathbb{R}}}
\newcommand{\nat}{\ensuremath{\mathbb{N}}}
\newcommand{\cset}{\ensuremath{\mathcal{C}}}
% ========================== Ranges =========================
\newcommand{\rgeq}[1]{\ensuremath{_{\geq #1}}}
\newcommand{\rleq}[1]{\ensuremath{_{\leq #1}}}
\newcommand{\rg}[1]{\ensuremath{_{>#1}}}
\newcommand{\rl}[1]{\ensuremath{_{<#1}}}
\newcommand{\rii}[2]{\ensuremath{_{[#1,#2]}}}
\newcommand{\rie}[2]{\ensuremath{_{[#1,#2)}}}
\newcommand{\rei}[2]{\ensuremath{_{(#1,#2]}}}
\newcommand{\ree}[2]{\ensuremath{_{(#1,#2)}}}
\newcommand{\elrii}[2]{\ensuremath{_{\{#1,#2\}}}}
% ===================== Constraint sets =====================
% \_con ----- Produces the constraint set for the indicated quantity. One optional arg adds a substcript.
\NewDocumentCommand{\xcon}{ o o }{%
	\IfNoValueTF{#2}%
	{		
	\IfNoValueTF{#1}%
		{\ensuremath{\mathcal{X}}}%
		{\ensuremath{\mathcal{X}_{#1}}}}
		{\ensuremath{\mathcal{X}{(#1,#2)}}}}
\NewDocumentCommand{\ucon}{ o }{%
	\IfNoValueTF{#1}%
		{\ensuremath{\mathcal{U}}}%
		{\ensuremath{\mathcal{U}_{#1}}}}
\NewDocumentCommand{\tcon}{ o }{%
	\IfNoValueTF{#1}%
		{\ensuremath{\mathcal{T}}}%
		{\ensuremath{\mathcal{T}_{#1}}}}
\NewDocumentCommand{\wcon}{ o }{%
	\IfNoValueTF{#1}%
		{\ensuremath{\mathcal{W}}}%
		{\ensuremath{\mathcal{W}_{#1}}}}

% ==================== Switching signals ====================
% \ss --------- General switching signal
% \sucset ----- Successor matrix associated with signal #1. Ex. \sucmat{\dyn\ss}
% \sucsetel --- Successor matrix element (#2,#3) associated with signal #1. Ex. \sucset{\dyn\ss}{\dyn\modeidx_i}{\dyn\modeidx_j}

% \mindts ----- Vector of the mode-dependent min-DTs for signal #1. Ex \mindts{\dst\ss}
% \maxdts ----- Vector of the mode-dependent max-DTs for signal #1. Ex \maxdts{\dst\ss}
% \mindt ------ The mode-dependent min-DT for signal #1, mode #2. Ex \mindt{\dst\ss}{\dst\modeidx}
% \maxdt ------ The mode-dependent max-DT for signal #1, mode #2. Ex \maxdt{\dst\ss}{\dst\modeidx}
% \sucsetcon -- Succesoor matrix constraint

% \mindtscon -- Vector of min-dt constraints
% \maxdtscon -- Vector of max-dt constraints
% \mindtcon --- Single min-dt constraint
% \maxdtcon --- Single max-dt constraint
% \ssset     -- Set of switching signals that respects up to three constraint elements

% \sstimer   -- Ss timer. Maps the time and a ss to the number of steps since a switch that new information was gleaned
% \sstimerval - A Value of switching signal timer
% \sstimermax - Maximum value of the switching signal timer in mode #1
% \sspairset -- The set of possible pairs given a mode #1 and timer value #2
% \sscount ---- The number of times the switching signal has changed at some time t

\renewcommand{\ss}{\ensuremath{{\sigma}}}
\newcommand{\sucset}[1]{\ensuremath{\mathcal{S}^{#1}}}
\newcommand{\sucsetrow}[2]{\ensuremath{\sucset{#1}_{#2}}}
\newcommand{\sucsetel}[3]{\ensuremath{{s_{#2,#3}^{#1}}}}

\newcommand{\mindts}[1]{\ensuremath{\underline{\mathcal{D}}^{#1}}}
\newcommand{\maxdts}[1]{\ensuremath{\overline{\mathcal{D}}^{#1}}}
\newcommand{\mindt}[2]{\ensuremath{{\underline{\delta}_{#2}^{#1}}}}
\newcommand{\maxdt}[2]{\ensuremath{{\overline{\delta}_{#2}^{#1}}}}
\newcommand{\sucsetcon}{\ensuremath{{S}}}
\newcommand{\sucsetconrow}[1]{\ensuremath{{S_{#1}}}}

\newcommand{\mindtscon}{\ensuremath{\underline{D}}}
\newcommand{\maxdtscon}{\ensuremath{\overline{D}}}
\newcommand{\mindtcon}[1]{\ensuremath{{\underline{d}_{#1}}}}
\newcommand{\maxdtcon}[1]{\ensuremath{{\overline{d}_{#1}}}}
%\newcommand{\ssset}[3]{\ensuremath{\Sigma(#1,#2,#3)}}
\NewDocumentCommand{\ssset}{ o o o }{%
	\IfNoValueTF{#3}%
		{\IfNoValueTF{#2} %no 3
			{\IfNoValueTF{#1}%no 2, no 3
				{\ensuremath{\Sigma}}
				{\ensuremath{\Sigma(#1)}}
			}
			{\ensuremath{\Sigma(#1,#2)}}%no 3 but 1 and 2
		}
		{\ensuremath{\Sigma(#1,#2,#3)}}}
		
\newcommand{\sstimer}{\ensuremath{{T}}}
\newcommand{\sstimerval}{\ensuremath{\tau}}
\newcommand{\sstimermax}[1]{\ensuremath{{\overline{\sstimerval}_{#1}}}}
\newcommand{\sspairset}[2]{\ensuremath{{\Lambda((#1,#2))}}}
\newcommand{\sscount}{\ensuremath{{\ss^\#}}}

\newcommand{\switchtimes}[1]{\ensuremath{\tau^{#1}}}
\newcommand{\switchtime}[2]{\ensuremath{{\switchtimes{#1}_{#2}}}}
\newcommand{\lastswitchtime}[1]{\switchtimes{#1}_+}

% ================== System Modes ==================
% \modes -------- A collection of modes
% \mode --------- A single mode indexed by #1
% \nummodes --- Number of modes 
% \modeidx ------ Index of a single mode \ss(t)=\modeidx

% \nx ------------- The state dimension
% \nu ------------- The input dimension

\newcommand{\modes}{\ensuremath{\mathcal{M}}}
\newcommand{\mode}[1]{\ensuremath{\modes_{#1}}}
\newcommand{\nummodes}{\ensuremath{M}}
\newcommand{\modeidx}{{\ensuremath{\mu}}}

\newcommand{\nx}{\ensuremath{{n_x}}}
\renewcommand{\nu}{\ensuremath{{n_u}}}
\NewDocumentCommand{\futstate}{ o o o o}{%
	\IfNoValueTF{#3}%
		{\ensuremath{x{(#1,#2)}}}
		{\ensuremath{\phi(#1;#2,#3,#4)}}}
\NewDocumentCommand{\useq}{}{\ensuremath{\mathfrak{u}}}
\NewDocumentCommand{\feasuseq}{mmm}{%
	\ensuremath{\mathfrak{U}(#1,#2,#3)}
	}
	
% ==================== Safe sets ====================
\newcommand{\safesets}{\ensuremath{\mathcal{S}}}
\newcommand{\safeset}[2]{\ensuremath{\safesets_{(#1,#2)}}}




% ==================== Misc ====================
\newcommand{\narroweq}{\ensuremath{\scalerel[1ex]{=}{\phantom{=}\hspace{-0.17cm}}}}