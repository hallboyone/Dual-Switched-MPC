\section{System Description}
Consider a collection of $\numagents$ external switching signals each respecting their own directed graph, $\ss_\agentidx(t)\in\Sigma(\mathcal{G}_\agentidx)\ \agentidx\in\agentidxset\triangleq\int\rii{1}{\numagents}$. Then the dynamics of the system being studied take the following form
\begin{align}
A(t)&=\begin{bmatrix}
A^{11}_{\ss_1(t)}&A^{12}_{\ss_1(t)}&\cdots&A^{1\numagents}_{\ss_1(t)},\\
A^{21}_{\ss_2(t)}&A^{22}_{\ss_2(t)}&\cdots&A^{2\numagents}_{\ss_2(t)},\\
\vdots&\vdots & \ddots & \vdots\\
A^{\numagents 1}_{\ss_\numagents(t)} & A^{\numagents 2}_{\ss_\numagents(t)} &\cdots & A^{\numagents \numagents}_{\ss_\numagents(t)} 
\end{bmatrix},\\
B(t)&=\begin{bmatrix}
B^{1}_{\ss_1(t)}&0&\cdots&0,\\
0&B^{2}_{\ss_2(t)}&\cdots&0,\\
\vdots&\vdots & \ddots & \vdots\\
0 & 0 &\cdots & B^{\numagents}_{\ss_\numagents(t)} 
\end{bmatrix}.
\end{align}
This structure is motivated by dynamically coupled distributed systems. Note how each block-row is governed by a single switching signal. This makes intuitive sense because local switching signals are more likely to effect how neighboring states impact the local agent rather then how local states will effect neighboring agents.

Our objective is to design safe-set collections for systems with the above structure. Recall that every element in a safe-set collection must be within the one-step preset of all safe-set collection elements indexed by the possible states of the switching signal after a single time step. The structure of the system under consideration makes this an especially challenging prospect for two reasons. First, the number of successor safe-sets grows exponentially when there are two or more independent switching signals. Second, systems that take this form would tend to be larger than trivial examples. These suggest that previous set based techniques will struggle due to poor scaling in both dimension and the number of modes. These concerns will be addressed by splitting the system into block-rows and looking for safe-set collections for each separately. Once found, the collections can be merged into a large collection with all possible switching signal states represented.

\subsection{Block-row Dynamics}
Looking only at the block row indexed by $\agentidx$, the dynamics can be rewritten as 
\begin{align}\label{eq:block-row-dyn}
x_\agentidx(t+1)=A^{\agentidx\agentidx}_{\ss_\agentidx(t)}x_\agentidx(t)&+B^\agentidx_{\ss_\agentidx(t)}u_\agentidx(t)\nonumber\\&+\sum_{\tilde\agentidx\in\agentidxset\setminus \agentidx}A^{\agentidx\tilde\agentidx}_{\ss_\agentidx(t)}x_{\tilde\agentidx(t)}.
\end{align}
This system has only a single switching signal explicitly appearing, $\ss_\agentidx(t)$ and resembles a locally switched system with additive disturbances. Previous, robust switched techniques, such as those developed in \cite{Lavaei2021}, may seem like possible solutions. However, these previous techniques rely on bounded additive disturbances and the set that $x_{\tilde{\agentidx}}$ is drawn from is not obvious. The full state constraints could be used but this would lead to a conservative result. Alternatively, the current safe-set containing $x_{\tilde{\agentidx}}$ could be used. This however, returns the system to the centralized problem with its associated drawback. Furthermore, if \autoref{eq:block-row-dyn} represents a distributed system, this level of communication may be undesirable. Balancing these considerations, this work uses the convex hull of the safe-set collection union. This only requires acquiring a single set and does not rely on the state of any switching signal except the local one. 

A circular dependency arose in the previous discussion -- the safe-set collection for $x_\agentidx$ depend on the collection for $x_{\tilde{\agentidx}}$ which depends on the collection for $x_\agentidx$. This suggests an iterative algorithm with parallel elements. This is developed further in the following section. 