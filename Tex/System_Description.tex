%\begin{figure*}[t]
%\centering
%\includegraphics[width=\textwidth]{./figures/sample_system}
%\end{figure*}

\section{System Description}
In the previous sections, the systems have only contained a single switching signal. In general, however, switching signals may come from independent sources and merging them into a single signal would not be justified. Consider a collection of $\numagents\in\int\rgeq{1}$ external switching signals each respecting their own directed graph, $\ssl[\agentidx](t)\in\Sigma(\graph^\agentidx)\ \agentidx\in\idxset{\numagents}$. Further, consider a system with all of these signals and dynamics taking the form
\begin{subequations}
\label{eq:sys}
\begin{align}
A(t)&=\begin{bmatrix}
A^{11}_{\ssl[1](t)}&A^{12}_{\ssl[1](t)}&\cdots&A^{1\numagents}_{\ssl[1](t)},\\
A^{21}_{\ssl[2](t)}&A^{22}_{\ssl[2](t)}&\cdots&A^{2\numagents}_{\ssl[2](t)},\\
\vdots&\vdots & \ddots & \vdots\\
A^{\numagents 1}_{\ssl[\numagents](t)} & A^{\numagents 2}_{\ssl[\numagents](t)} &\cdots & A^{\numagents \numagents}_{\ssl[\numagents](t)} 
\end{bmatrix},\label{eq:sys_A}\\
B(t)&=\begin{bmatrix}
B^{1}_{\ssl[1](t)} & \cdots & 0\\
\vdots             & \ddots & \vdots\\
0                  & \cdots & B^{\numagents}_{\ssl[\numagents](t)}
\end{bmatrix},\label{eq:sys_B}\\
\xcon(t)&=\xcon[\ssl[1](t)]^1\times\cdots\times\xcon[\ssl[\numagents](t)]^\numagents,\label{eq:sys_X}\\
\ucon(t)&=\ucon[\ssl[1](t)]^1\times\cdots\times\ucon[\ssl[\numagents](t)]^\numagents\label{eq:sys_Y}.
\end{align}
\end{subequations}
This structure is motivated by distributed systems with coupling in the system's states. Note how each block-row is governed by a single switching signal. This makes intuitive sense because local switching signals are more likely to effect how neighboring states impact the local agent rather then how local states will effect neighboring agents.

The objective of this work is to design a safe-set collection for systems with the above structure. As mentioned in the introduction, a single signal could be defined that amalgamates all the individual switching signals. It can now be seen, however, that corresponding to the exponential growth in the number of signal states is an exponential growth in the cardinality of the safe-set collection. This is computationally untenable. Furthermore, systems taking this for will likely be in a high dimension Having even just three block-rows of three dimensions each leads to a nine dimension system and will tax this naive implementation. These concerns will be addressed by splitting the system into block-rows and looking for safe-set collections, $\mathcal{S}^{\agentidx}=\{\mathcal{S}^\agentidx_n\}_{n\in\gnumnodes[\agentidx]}$, for each separately. Once found, the collections can be merged into a large collection with all possible switching signal states represented.

\subsection{Block-row Dynamics}
Looking only at the block row indexed by $\agentidx$, the dynamics can be rewritten as 
\begin{align}\label{eq:block-row-dyn}
x_\agentidx(t+1)&=A^{\agentidx\agentidx}_{\ssl[\agentidx](t)}x_\agentidx(t)+B^\agentidx_{\ssl[\agentidx](t)}u_\agentidx(t)\nonumber\\&\quad+\sum_{\tilde\agentidx\in\idxset{\numagents}\setminus \agentidx}A^{\agentidx\tilde\agentidx}_{\ssl[\agentidx](t)}x_{\tilde\agentidx}(t)\\
&=A^{\agentidx}_{\ssl[\agentidx](t)}x_\agentidx(t)+B^\agentidx_{\ssl[\agentidx](t)}u_\agentidx(t)+E_{\ssl[\agentidx](t)}^\agentidx w(t),\nonumber
\end{align}
with $w(t)\in\wcon[\ssl[\agentidx](t)]^\agentidx$. The existence of $\wcon[\ssl[\agentidx](t)]^\agentidx$ can be inferred from the fact that the full system is constrained. These dynamics and constraints are collected into the following tuple defining agent $\agentidx$,
$$\agent{\agentidx}\triangleq\{\{A^{\agentidx}_{\modeidx},B^\agentidx_{\modeidx}, E_{\modeidx}^\agentidx,\xcon[\modeidx]^\agentidx,\ucon[\modeidx]^\agentidx, \wcon[\modeidx]^\agentidx\}_{\modeidx=1}^{\nummodes[\agentidx]},\graph^\agentidx\}.$$
These can be collected into the full system
\begin{equation}\label{eq:agent_notation}
\agents\triangleq\{\agent{\agentidx}\}_{\agentidx\in\idxset{\numagents}}.
\end{equation}
Each agent has only has a single switching signal explicitly appearing, $\ss_\agentidx(t)$, and resembles a locally switched system with additive disturbances. Previous, robust switched techniques, such as those developed in \cite{Lavaei2021}, may seem like possible solutions. However, though $\wcon[\modeidx]^\agentidx$ can be characterized using the full state constraints, this would lead to conservative results because the actual safe-sets are, by definition, subsets of the state constraints. Alternatively, each of the neighboring agents could share its current safe-set. This however, returns the system to the centralized problem with its associated drawbacks. Furthermore, if \autoref{eq:block-row-dyn} represents a distributed system, this level of communication may be undesirable. Balancing these considerations, this work bounds the states of any given neighbor within the convex hull of the union of that neighbor's safe-set collection. This only requires acquiring a single set and relies on the state of the local switching signal alone. 