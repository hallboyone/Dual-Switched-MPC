\section{Algorithm Design}
We begin with an algorithm to find the safe-sets of a single agent, $\agent{\agentidx}$, assuming the the other agents' safe-sets are fixed. Let $\mathcal{W}_{\tilde{\agentidx}}$, $\tilde{\agentidx}\in\int\rii{1}{\numagents}\setminus\agentidx$ be the convex hull of the union of the safe-sets for each of the other agents and define 
\begin{equation}
\wcon[\agentidx][\modeidx]=\bigoplus_{\tilde{\agentidx}\in\int\rii{1}{\numagents}\setminus\agentidx}A_\modeidx^{\agentidx,\tilde{\agentidx}}\mathcal{W}_{\tilde{\agentidx}}.
\end{equation}
This represents the set of additive errors agent $\agent{\agentidx}$ can experience while in mode $\modeidx$. Denote the collection of these sets as $\wconset{\agentidx}=\{\wcon[\agentidx][\modeidx]\}_{\modeidx=1}^{\nummodes[\agentidx]}$. With these definitions, the following algorithm is introduced. 
\begin{algorithm}[t]
\caption{Nodal safe-sets with previewed disturbances}\label{alg:node_safe_sets}
\begin{algorithmic}[1]
\Procedure {NodalSafeSets}{$\agent{\agentidx}$, $\wcon$}
\State $k\gets0$
\State $\Omega_{(\modeidx,\sstimerval)}^k\gets\xcon[\modeidx]$ for all $\modeidx\in\int\rii{1}{\nummodes[\agentidx]},\ \sstimerval\in\int\rii{1}{\sstimermax{\modeidx}}$.
\Repeat 
	\State $k\gets k+1$
	\For{$\modeidx\in\int\rii{1}{\nummodes[\agentidx]}$}
		\For{$\sstimerval\in\int\rii{1}{\sstimermax{\modeidx}}$}
			\State $\Omega_{(\modeidx,\sstimerval)}^k\gets\Omega_{(\modeidx,\sstimerval)}^{k-1}$
			\For{$(\tilde{\modeidx},\tilde{\sstimerval})\in\sspairset{\modeidx}{\sstimerval}$}
				\State $\Omega_{(\modeidx,\sstimerval)}^k\gets\Omega_{(\modeidx,\sstimerval)}^k\cap\PreviewedPre[\tilde{\modeidx}][1]{\Omega_{(\tilde{\modeidx},\tilde{\sstimerval})}^{k-1}, \wcon[\agentidx][\modeidx]}$
			\EndFor
		\EndFor
	\EndFor
\Until{$\Omega_{(\modeidx,\sstimerval)}^k=\Omega_{(\modeidx,\sstimerval)}^{k-1}\ \forall\ \modeidx\in\int\rii{1}{\nummodes[\agentidx]},\ \sstimerval\in\int\rii{1}{\sstimermax{\modeidx}}$}
$\safeset{\modeidx}{\sstimerval}\gets\Omega_{(\modeidx,\sstimerval)}^k$ for all $\modeidx\in\int\rii{1}{\nummodes[\agentidx]},\ \sstimerval\in\int\rii{1}{\sstimermax{\modeidx}}$.\;
\EndProcedure
\end{algorithmic}
\end{algorithm}


\begin{algorithm}[t]
\caption{Distributed safe-set collection}\label{alg:safe_sets}
\begin{algorithmic}[1]
\Procedure {SystemSafeSets}{$\agents$}
\State $\Omega^0\gets\{\{\{\underline{0}\}_{\tau\in\mathcal{I}_\tau(\agentidx,\modeidx)}\}_{\modeidx\in\mathcal{I}_\modeidx(\agentidx)}\}_{\agentidx\in\mathcal{I}_\agentidx}$
\State $\Phi^0\gets\{\underline{0}\}_{\agentidx\in\mathcal{I}_\agentidx}$
\State $k\gets0$
\Repeat 
	\ParFor{$\agentidx\in\mathcal{I}_\agentidx$}
		\State $\Omega^{k+0.5}_{(\agentidx)}\gets\Call{NodeSafeSets}{\modes[\agentidx], \Phi^k}$
		\State $\Phi^{k+0.5}_{(\agentidx)}\gets\Call{ConHull}{\bigcup\Omega^{k+0.5}_{(\agentidx)}}$
	\EndParFor
	\ParFor{$\agentidx\in\mathcal{I}_\agentidx$}
		\State $\Omega^{k+1}_{(\agentidx)}\gets\Call{NodeSafeSets}{\modes[\agentidx], \Phi^{k+0.5}}$
		\State $\Phi^{k+1}_{(\agentidx)}\gets\Call{ConHull}{\bigcup\Omega^{k+1}_{(\agentidx)}}$
	\EndParFor
	\State $k\gets k+1$
\Until{$\Omega_{(\agentidx)}^k=\Omega_{(\agentidx)}^{k-1}\ \forall\ \agentidx\in\mathcal{I}_\agentidx$}

\State $\safesets\gets\Omega^k$
\EndProcedure
\end{algorithmic}
\end{algorithm}

\begin{lemma}
Given any agent, $\agent{\agentidx}$, and constraint set $\wcon$, $\Call{AgentSafeSets}{\agent{\agentidx},\wcon}$ returns the maximal safe-set collection.
\end{lemma}
\begin{proof}
Follows the logic of \cite[Theorem 2]{Danielson2019} but with previewed preset operations.
\end{proof}
\begin{lemma}
Given an agent, $\agent{\agentidx}$, and the sets $\hat\wcon\subseteq\tilde\wcon$, the relationship{\small
$$\Call{AgentSafeSets}{\agent{\agentidx},\hat\wcon}\supseteq\Call{AgentSafeSets}{\agent{\agentidx},\tilde\wcon}$$}
holds element-wise. 
\end{lemma}
\begin{proof}

\end{proof}
\begin{lemma}
The algorithm $\Call{SystemSafeSets}{\agents}$ produces a valid safe-set collection for every $k\in\int\rgeq{0}$. 
\end{lemma}
\begin{proof}
This holds trivially for $\Omega^0$. The following induction steps complete the proof.
\begin{enumerate}
	\item Assume $\Omega^k$ is valid under the disturbance constraints $\Phi^k$.
	\item By Lemma 1, $\Omega^k\subseteq\Omega^{k+0.5}$.
	\item By Lemma 2, $\Phi^k\subseteq\Phi^{k+0.5}$ implies that $\Omega^{k+1}\subseteq\Omega^{k+0.5}$.
	\item Since $\Omega^{k+1}$ is valid for $\Phi^{k+0.5}$, it will also be valid for $\Phi^{k+1}\subseteq\Phi^{k+0.5}$.
\end{enumerate}
\end{proof}