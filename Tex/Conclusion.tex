\section{Conclusion}
In the previous literature ensuring persistent feasibility in externally switched systems, the curse of dimensional has prevented the study of large scale systems. Furthermore, previous methods have been limited to consider only single switching signals. This work addresses both of these difficulties in systems representing distributed systems coupled through the states. An graph based constraint scheme was described to constrain each agent's switching signal that provides much greater flexibility than previous dwell time constraints. Iterative algorithms where then developed that efficiently computed time-varying state constraints that, when respected, ensure feasibility under all permutations of the switching signals. 

Though this work provides techniques of ensuring feasibility, it makes limited use of results from distributed system literature. Currently, only the current state of neighboring agents is used. In future work, a unified framework will be developed that uses the multi-step state prediction provided by receding horizon controllers will be used to loosen the bounds on feasibility and to reduce the cost of running the system. Further, future work should examine advancements allowing for plug-and-play features that would allow agents to be dynamically added and removed from the system. 