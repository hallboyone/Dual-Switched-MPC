\section{Introduction}
Hybrid systems are a broad and important class of systems with both discrete and continuous (or discrete approximations of continuous) dynamics. A subclass of hybrid systems are those whose discrete dynamics are purely time dependent and unknown to the system. For example, user input and component failures can both \edit{cause the system to suddenly change}{be modeled as a change in a discrete variable that is independent of the continuous state and unforeseeable before hand}. Such \edit{discrete, external dynamics}{externally-triggered events} can be viewed as an external signal switching the system's dynamics at unforeseen times. If the system is constrained, then special care should be taken to make the system robust to these switches so that state and input constraints are respected. 

% Lit review - Centralized Switched
If the system were allowed to be switched arbitrarily, then satisfying the constraints would require the controller to always keep the state within a control invariant set common to all modes \cite{Liberzon1999}. If the modes are not extremely similar, this may be a \edit{very}{} difficult or impossible requirement meet. Instead of arbitrary switching, \edit{dwell time and successor constraints are often imposed on the switching signal \cite{Liberzon1999,Morse1999}.}{various constraints on the switching signal have been suggested in previous literature. Graph based constraints use directed graphs to constrain how a switching signal may evolve at each time step \cite{Santis2004, Athanasopoulos2014, Athanasopoulos2017, Fiacchini2018}} Minimum dwell time constraints give the system time to recover after a switch and prepare for the next one \cite{Danielson2019}, while maximum dwell time constraints ensure that the system will not dwell in ``poorly'' behaved modes for too long \cite{Hall2022}. These constraints can either be generated by the physical implementation and the controller designed to accommodate them \cite{Danielson2019}, or they can be specified by the controller and enforced in the physical implementation \cite{Zhang2016}. Dwell time and successor constraints have been used in previous works to create time-varying control invariant (CI) sets that are robust to all possible switching signals \edit{}{\cite{Danielson2019, Athanasopoulos2017}}. Unfortunately, these methods can be very computationally expensive and suffer greatly from the curse of dimensionality, precluding their use in large-scale systems. To address this, approximate set-based methods have been used to reduce the computational complexity \edit{}{\cite{Santis2004,Athanasopoulos2017}}. No previous solution, however, utilized the structure of the system to reduce the computational load of ensuring persistent feasibility. 

% Lit review - Multiple switching signals
A further shortcoming of the current literature is that only a single switching signal is explicitly considered. A naive solution would simply be to define a new switching signal with modes corresponding to all the realizable permutations of the independent signals. While this works in theory, it creates an exponential growth in the number of possible system modes. Enumerating multiple switching signals and defining the dwell time and successor constraints of each mode quickly leads back to arbitrary switching and its associated challenges. Even if constraints could be maintained, the increase in the computational load with be difficult to overcome. 

A better solution for a system with multiple switching signals is to divide it into discrete parts and examine each separately. Variations of this goal have been addressed in several previous works. Systems with a similar structure where studied in \cite{Riverso2015} which examined both the constrained and unconstrained cases. This was done, however, without any external switching, instead focusing on plug-and-play design objectives. Other works examined time-varying state constraints for decentralized systems where shifting communication topologies introduce external switching in \cite{Ahandani2020, Li2020}. In \cite{Li2020}, individual agents collaborated by sharing a preview of their future trajectories, inspiring some of the strategies here. However, unlike \cite{Ahandani2020, Li2020}, we consider changing local dynamics, and allow additional state constraints compared to \cite{Li2020}.

%The study of feasibility in externally switched systems has been addressed in a mostly separate field of study. The general method of ensuring persistent feasibility has relied on the construction of time-varying control invariant sets. \cite{Santis2004, Zhang2016, Danielson2019}. To address the problem of computational intractability, some previous works, such as \cite{Santis2004}, have used approximate set-based methods to reduce the computational complexity or terminate the algorithms early. None have, however, utilized the structure of the system to reduce the computational load of ensuring persistent feasibility. 

This work develops an algorithm to find time-varying state constraints that ensure persistent feasibility for systems with multiple sources of switching. The algorithm's most computationally expensive steps are parallelizable, greatly improving its scalability. 

In the next section, notation and concepts are introduced that will be used throughout the reminder of the paper. Next, the general form of the system is described and the associated challenges are discussed further. Finally, several algorithms are introduced as key contributions of this work and their theoretical properties and implementation are explored. These are then applied to a numerical example that demonstrates their effectiveness. 