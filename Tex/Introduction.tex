\section{Introduction}
Hybrid systems are a broad and important class of systems with both discrete and continuous (or discrete approximations of continuous) dynamics. A subclass of these are systems whose discrete dynamics are purely time dependent and unknown to the system. Such discrete, external dynamics can be viewed as an external signal switching the system dynamics at discrete, unforeseen times. 

Many systems experience changes to its dynamics beyond the input of the controller. For example, user input and component failures can both cause the system to suddenly change. If the system is constrained, then special care should be taken to make the system robust to these switches so that constraints are not violated. 

If the system were allowed to be switched arbitrarily, then satisfying the constraints would be restrictive indeed. This would require the controller to always keep the state within a control invariant set common to every modes. If the modes are not extremely similar, this may be a very difficult requirement meet. 

Instead of arbitrary switching, dwell time and successor constraints are often imposed on the switching signal. Minimum dwell time constraints give the system time to recover after a switch and prepare for the next one while maximum dwell time constraints ensure that the system will not dwell in ``poorly'' behaved modes for too long. These constraints can either be generated by the physical implementation and used by the controller or generated by the controller and enforced in the physical implementation.

Dwell time and successor constraints have been used in previous works to create time-varying control invariant sets that are robust to all possible switching signals \alert{cite}. An issue with these previous methods is that they can be very computationally expensive and suffer greatly from the curse of dimensionality. This precludes their use in large systems. 

A further shortcoming of the current literature is that only a single switching signal is considered. It is not difficult to imagine cases where there are multiple sources of switching that are independent from each other. Consider a distributed system where the individual agents are switching according to local switching signals. This independence makes a common switching signal with shared dwell time and successor constraints much more difficult to motivate. Seeking to enumerate the possible states of even just two switching signals and define their dwell time and successor constraints leads quickly back to arbitrary switching but with many more modes.

This work develops an algorithm with applications to both reduce the computational expense of previous algorithms and analyze of systems with multiple sources of switching. The algorithm's most computationally expensive steps are parallelizable, greatly improving its scalability. In the next section, notation and concepts are introduced that will be used throughout the reminder of the paper. Next, the general form of the system is described and the associated challenges are discussed further. Finally, several algorithms are introduced as key contributions of this work and their theoretical properties and implementation are explored. These are then applied to a numerical example that demonstrates the effectiveness of the results. 

Related works includes \cite{Ahandani2020} where the authors examine decentralized systems under MPC with switching communication topologies and coupling in the constraints and inputs. The local agents are assumed LTI while the cross dynamics change according to the switching signal. This leads to an external disturbance applied to each node that is rejected using tube-based, switched MPC.