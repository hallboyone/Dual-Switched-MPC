\documentclass{article}
\usepackage[utf8]{inputenc}
\usepackage[margin=0.75in]{geometry}
\usepackage{amsmath}
\usepackage{amssymb}
\begin{document}
% The numerical example focuses on a simple micro-grid system with its own power source, battery bank, local load, and links to the larger grid. The controller's objective is twofold, minimize the wear on the system's components while selling as much power back to the grid as possible. In addition, the system is constrained to meet the local load at all times.

% The solar array generates power at a rate of $P_{s}(t)$. The value of $P_s(t)$ is available as a forecast over the next $N$ time steps. 

Lets assume that there is a black box system that produces a forecast of the net power from local loads and sources over the control horizon. Denote this forecast as $P^\Delta(t)=\{P^\Delta_0(t),...,P^\Delta_{N-1}(t)\}\in\{\mathbb{R}\times\cdots\times\mathbb{R}\}$. Note that this work assumes the preview evolves according to some nominal, linear dynamics. Obviously, there is no underlying linear system that generates the forecast $P^\Delta(\cdot)$. Instead, we make use of the following observations/assumptions
\begin{enumerate}
    \item All linear systems behave the same way when at the origin.
    \item The general shape of the forecast, called the power-delta curve, will be the same each day. It will include a relatively flat night-time region, a morning region where the surplus power increases, and an evening region where the surplus power decreases back to the nighttime levels.
\end{enumerate}

To begin, assume the nominal nighttime power shortage is given by $\hat{P}^{\Delta_n}$ and shift the forecast at each time step by this amount. Since the night will be when the power shortage is largest, this shift will create a very rough curve that is near zero during the night, increases to a maximum over the course of the morning, and then decreases back to near zero during the evening. Using the observations that any stable linear system will track a near-zero signal relatively well, this justifies the use of linear systems for this region of the power-delta curve. 

Turning now to the dynamic portion of the power-delta curve, suppose we fit some $2^{nd}$ order state space system that will track this curve. As mentioned above, there isn't a linear system under these dynamics so this may be viewed as an abuse of mathematics. However, the curve can be expected to be relatively smooth and with a predictable period based on the length of the day. This coupled with the fact that system does not need to track the actual preview well justifies the use of the state space system fit to some nominal values.

The resulting system, lets call it $\amthcal{H}$, will be able to track both the daytime and nighttime portions of the power-delta curve well enough. The transitions, however, could possible be troublesome. 
\end{document}